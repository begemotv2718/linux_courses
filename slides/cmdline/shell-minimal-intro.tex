\begin{frame}[fragile]{Оболочка операционной системы}

     \begin{block}{Оболочка операционной системы}
     (от англ. shell «оболочка») \alert{интерпретатор команд} операционной системы, обеспечивающий интерфейс для взаимодействия пользователя с функциями системы.  
     \end{block}
     \only<1>{
     Что такое Unix shell?
     \begin{itemize}
        \item Обычная программа, запускающаяся после входа в систему
        \item Интерактивный командный интерпретатор
        \item Платформа интеграции (для утилит)
        \item Язык программирования
     \end{itemize}
   }
   \only<2>{
	\vspace{0.5in}
	Пример shell из  Windows-world --  cmd.exe, PowerShell
	\vspace{0.5in}
	Минимальный дистрибутив Linux -- ядро + shell 
   }
\end{frame}
\note{   
     Как вы помните задачи ядра - управление процессами и потоками, управление
     памятью, Управление файлами, Разграничение доступа, мониторинг и
     конфигурация. Shell обеспечивает пользователю  интерфейс доступа к этим функциям ядра.    
     Как Интерактивный командный интерпретатор shell предосталяет средства:
     редактора командной строки, часто со средствами упрощающими ввод команд
     такими как  горячие клавишами, автодополнение, история команд и поиск по
     истории, сокращения команд (aliases), сокращения путей к файлам c помощью
     спецсимволов. 
     Как Платформа интеграции предоставляет средства пренаправления
     ввода/вывода, каналы (pipes), и возможность создавать скрипты.
     Одновременно является языком программирования, но эту часть будем
     рассматривать позже. 
}


\begin{frame}[fragile]{Как работать с командной строкой}
  \begin{itemize}
    \item 
	Находим приглашение командной строки
	\$, \#, user@host:~\$
    \item
	Вводим имя команды, опции, аргументы, запускаем на выполнение нажатием <Enter>
   \end{itemize}

	Что такое команды?
  \begin{itemize}
    \item исполняемая программа (бинарный файл, скрипт)
    \item встроенные в оболочку команды (shell built-ins)
    \item функция оболочки
    \item сокращение команды (an alias) 
  \end{itemize}
\end{frame}
\note {
Система ожидает ввода команды, показывая приглашение командной строки.
Пользователь вводит команды опции аргументы и нажимает <Enter>. 
Опции - модификаторы поведения программы.  
Аргументы - то над чем производятся действия.
}

