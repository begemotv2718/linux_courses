\begin{frame}
  \frametitle{DeviceTree: что это такое?}
\textbf{DeviceTree} -- структура данных, содержащая информацию о конфигурации конкретной платы. 

\vspace{1cm}

Существует в двух модификациях -- человекочитаемая (файлы \texttt{*.dts}) и машиночитаемая (файлы \texttt{*.dtb})

\vspace{1cm}

Бинарный файл генерируется из dts файла device tree компилятором (dtc).

\vspace{1cm}

Бинарный файл devicetree загружается в память одновременно с ядром, и используется ядром для подбора драйверов и загрузки параметров драйверов.
\end{frame}

\begin{frame}
  \frametitle{DeviceTree: Зачем это нужно?}
  \begin{itemize}
     \item Меньше вариантов сборки ядра
     \item Возможность поддерживать новые платы без пересборки ядра
     \item Меньше коммитов в драйвера ядра 
     \item Приспособлены для автоматизации
  \end{itemize}
\end{frame}

\begin{frame}
  \frametitle{Родственные технологии}
  \begin{itemize}
      \item Внутриядерные структуры данных с описанием архитектуры платы (/arch/arm/mach-*) 
      \item UEFI
      \item ACPI (i386 only)
      \item OpenFirmware (PowerPC, Solaris) прямой предок
  \end{itemize}
\end{frame}

\begin{frame}[fragile]
  \frametitle{Пример devicetree файла}
  \begin{lstlisting}
/dts-v1/;
/ {
        soc {
		#address-cells = < 0x1 >;
		#size-cells = < 0x1 >;
              pic_3: pic@100 {
                      reg = < 0x100 0x20 >;
                      interrupt-controller;
              };
              uart: uart@020000 {
                  interrupt-parent = < &pic_3 >;
                  interrupt-parent-path =  &pic_3 ;
                  reg = <0x020000 0x100>;
              };
        };

};
  \end{lstlisting}

\end{frame}
\begin{frame}[fragile]
  \frametitle{Пример devicetree файла}
  \begin{lstlisting}[language=C]
/dts-v1/;
/include/ "meson6.dtsi"
/ {
	model = "Geniatech ATV1200";
	compatible = "geniatech,atv1200", "amlogic,meson6";
	aliases {
		serial0 =  &uart_AO;
	};
	memory {
		reg = <0x40000000 0x80000000>;
	};
};
&uart_AO {
	status = "okay";
};
&ethmac {
	status = "okay";
};
  \end{lstlisting}
\end{frame}

\begin{frame}[fragile]
  \frametitle{Компиляция и декомпиляция}
  \begin{center}
    \textbf{Компиляция}
  \end{center}
  \begin{lstlisting}[language=bash]
dtc -O dtb -o example_a.dtb example_a.dts 
  \end{lstlisting}
  \begin{center}
    \textbf{Декомпиляция}
  \end{center}
  \begin{lstlisting}[language=bash]
dtc -O dts -o example_a.dts example_a.dtb
  \end{lstlisting}
\end{frame}

\begin{frame}[fragile]
  \frametitle{Важные элементы синтаксиса}
  \begin{center}
    \textbf{Численные параметры}
  \end{center}
\begin{lstlisting}
#address-cells = <0x1> /* Одно число */
...
reg = <0x200000 0x20> /* Числовой массив */
\end{lstlisting}
\end{frame}
\begin{frame}[fragile]
   \begin{center}
     \textbf{Ссылки}
   \end{center}  
\begin{lstlisting}
 /{
   soc {
     uart1: uart@7e0205000 { ...}
     pic_1: pic@10000{}
   };
 };
   &uart1  /*Символьная ссылка*/
   {
     pic=<&pic_1>; /* Ссылка на phandle */
   }
\end{lstlisting}
\end{frame}

\begin{frame}
  \frametitle{Упражнение}
  \begin{itemize}
    \item Взять файл \texttt{example\_a.dts} и скомпилировать с помощью dtc
    \item Декомпилировать тот же файл
    \item Посмотреть, во что превратились ссылки на \texttt{pic\_3} 
  \end{itemize}
\end{frame}

\begin{frame}
  \frametitle{Наиболее интересные параметры в описании устройства}
  \begin{itemize}
    \item \texttt{compatible} -- строка, определяет выбор драйвера
    \item \texttt{status} -- строка, используется ли устройство \texttt{"ok"}/\texttt{"disabled"}
  \end{itemize}
\end{frame}

\begin{frame}
  \frametitle{Наиболее важные разделы DeviceTree}
  \begin{itemize}
    \item \textbf{cpus} Процессоры
    \item \textbf{memory} Структура памяти
    \item \textbf{aliases} Таблица символов
    \item \textbf{choosen} Параметры загрузки
  \end{itemize}
\end{frame}

\begin{frame}[fragile]
  \frametitle{Директива include и наследование}
  \begin{center}
    Внутренний синтаксис devicetree
  \end{center}
  \begin{lstlisting}
  /include/"filename.dtsi"
  \end{lstlisting}
  \begin{center}
    Использование C препроцессора
  \end{center}
  \begin{lstlisting}
  #include "filename.dtsi"
  \end{lstlisting}
  \begin{center}
    Наследование: При многократном определении одного и того же параметра, в итоговый devicetree подставляется последняя версия 
  \end{center}
\end{frame}

\begin{frame}[fragile]
  \frametitle{Компиляция при использовании C препроцессора}
\begin{lstlisting}[language=bash]
cpp -nostdinc -undef -D__DTS__ -x assembler-with-cpp \
 -Iarch/arm/boot/dts -Iarch/arm/boot/dts/include -o file.tmp mytree.dts
dtc -O dtb -o mytree.dtb -i arch/arm/boot/dts file.tmp 
\end{lstlisting}
\end{frame}

\begin{frame}
  \frametitle{Упражнение: Компиляция с include}
  \begin{itemize}
    \item Скомпилировать \texttt{example\_b.dts}
    \item Декомпилировать его же
    \item Посмотреть результат (какой параметр из dtsi переопределился?)
  \end{itemize}

\end{frame}

\begin{frame}[fragile]
  \frametitle{Оверлеи пример}
\begin{lstlisting}
// Enable the i2s interface
/dts-v1/;
/plugin/;

/ {
    compatible = "brcm,bcm2708";

    fragment@0 {
        target = <&i2s>;
        __overlay__ {
            status = "okay";
        };
    };
};
\end{lstlisting}
\end{frame}
\begin{frame}[fragile]
 \frametitle{Оверлеи}
\begin{lstlisting}[language=bash]
dtc -@ -I dts -O dtb -o patch.dtbo overlay.dts
dtoverlay patch.dtbo
dtoverlay -r patch.dtbo
\end{lstlisting}
\end{frame}
