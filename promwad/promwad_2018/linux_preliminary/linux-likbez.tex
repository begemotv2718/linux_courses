\documentclass[ignorenonframetext, professionalfonts, hyperref={pdftex, unicode}]{beamer}
%\usepackage{beamerthemesplit}

\geometry{paperwidth=140mm,paperheight=105mm}

%Hack to specify beamer folder https://tex.stackexchange.com/questions/275600/beamer-themes-on-custom-folder
\makeatletter
  \def\beamer@calltheme#1#2#3{%
    \def\beamer@themelist{#2}
    \@for\beamer@themename:=\beamer@themelist\do
    {\usepackage[{#1}]{\beamer@themelocation/#3\beamer@themename}}}

  \def\usefolder#1{
    \def\beamer@themelocation{#1}
  }
  \def\beamer@themelocation{}
\makeatother
%Packages to be included

\usepackage{graphicx}



\graphicspath{{./branding/}}
\usefolder{./branding}
\usetheme{Promwad}
%\usecolortheme{wolverine}


\usepackage[russian]{babel}
\usepackage[utf8]{inputenc}
\usepackage[T1]{fontenc}

%\usepackage[orientation=landscape, size=custom, width=16, height=9, scale=0.5]{beamerposter}

\usepackage{textcomp}


\usepackage{ulem}

\usepackage{verbatim}

\usepackage{ucs}


\usepackage{listings}
\lstloadlanguages{bash}

\lstset{escapechar=`,
	extendedchars=false,
	language=sh,
	frame=single,
	tabsize=2, 
	columns=fullflexible, 
%	basicstyle=\scriptsize,
	keywordstyle=\color{blue}, 
	commentstyle=\itshape\color{brown},
%	identifierstyle=\ttfamily, 
	stringstyle=\mdseries\color{green}, 
	showstringspaces=false, 
	numbers=none, 
%	numberstyle=\tiny, 
	breaklines=true, 
	inputencoding=utf8,
	keepspaces=true,
	morekeywords={u\_short, u\_char, u\_long, in\_addr}
	}

\definecolor{darkgreen}{cmyk}{0.7, 0, 1, 0.5}

\lstdefinelanguage{diff}
{
    morekeywords={+, -},
    sensitive=false,
    morecomment=[l]{//},
    morecomment=[s]{/*}{*/},
    morecomment=[l][\color{darkgreen}]{+},
    morecomment=[l][\color{red}]{-},
    morestring=[b]",
}

\author[Promwad]{{\bf Promwad}}

%\institution[EPAM]{EPAM}
%\logo{\includegraphics[width=1cm]{logo.png}}

\AtBeginSection[]{%
  \begin{frame}<beamer>
    \frametitle{}
    \tableofcontents[
        sectionstyle=show/shaded, hideallsubsections ]
  \end{frame}
  \addtocounter{framenumber}{-1}% If you don't want them to affect the slide number
}

\AtBeginSubsection[]{%
  \begin{frame}<beamer>
    \frametitle{}
    \tableofcontents[
        sectionstyle=show/hide,
        subsectionstyle=show/shaded/hide, ]
  \end{frame}
  \addtocounter{framenumber}{-1}% If you don't want them to affect the slide number
}


\title{Ликбез по Линуксу}

\begin{document}

\begin{frame}
  \frametitle{}
  \titlepage
\end{frame}

\section{Принципы проектирования переносимых программ}
\mode<all>{\begin{frame}{Главные ориентиры}
	\begin{itemize}
		\item кроссплатформенная переносимость
		\item открытые стандарты
	\end{itemize}
\end{frame}

\begin{frame}{Немного цитат}
Дуг Макилрой, изобретатель каналов <<pipes>>, сформулировал несколько постулатов,применимых для разработки ПО:
\pause
	\begin{itemize}
		\item пишите программы,  которые выполняют одну функцию и делают это хорошо;
			\pause
		\item пишите программы,  которые будут работать вместе;
			\pause
		\item пишите программы,  поддерживающие текстовые потоки,  поскольку они являются универсальным интерфейсом.
	\end{itemize}

\end{frame}


\begin{frame}{"Философия" UNIX}
	это {\bfseries не} философия,  а общие рекомендации по проектированию ПО,  накопленные сообществом программистов на опыте десятилетий разработок программ,  которые взаимодействуют друг с другом.
\end{frame}

\begin{frame}{1. Правило модульности}
	\begin{block}{Следует писать простые части,  связанные ясными интерфейсами.}
Единственным способом создания сложной программы,  не обреченной заранее на провал,  является сдерживание ее глобальной сложности.
	\end{block}
Т.е. построение программы из простых частей, соединенных четко определенными интерфейсами, 
так что большинство проблем являются локальными, 
и тогда можно рассчитывать на обновление одной из частей без разрушения целого.
\end{frame}

\begin{frame}{Размер кода и ошибки}
	\begin{center}
		\includegraphics[width=200px]{../../slides/intro/errors_density-graph.png}
	\end{center}
\end{frame}

\begin{frame}{2. Правило ясности}
	\begin{block}{Ясность -- лучше чем мастерство.}
		Последующее обслуживание программы -- важная и дорогостоящая часть жизненного цикла программы.
	\end{block}
	\pause
Писать программы необходимо так,  как если бы вы знали,  что последующей поддержкой будет заниматься неуравновешенный псих с топором,  знающий ваш домашний адрес!
\end{frame}

\begin{frame}{3. Правило композиции}
	\begin{block}{Следует разрабатывать программы,  которые будут взаимодействовать с другими программами.}
		Если разрабатываемые программы не способны взаимодействовать друг с другом,  то очень трудно избежать создания сложных монолитных  программ.
	\end{block}
	Методы взаимодействия могут быть сильными и слабыми -- по возможности рекомендуется использовать слабые методы и текстовые форматы передачи данных.
\end{frame}

\begin{frame}{4. Правило разделения}
	\begin{block}{Следует отделять политику от механизма и интерфейсы от основных модулей (engine).}
		Примеры политики и механизма:\\
		вид GUI и операции отрисовки, клиент (front-end) -- сервер (back-end), сценарии и библиотеки и др.
	\end{block}
	При жесткой связи политики и механизма:
	\begin{itemize}
		\item политика становится негибкой и усложняется ее изменение;
		\item изменение политики имеет строгую тенденцию к дестабилизации механизмов.
	\end{itemize}
\end{frame}

\begin{frame}{5. Правило простоты}
	\begin{block}{Необходимо проектировать простые программы и <<добавлять сложность>> только там,  где это необходимо.}
	\end{block}
	Основные причины добавления сложности:
	\begin{itemize}
		\item человеческий фактор (часто -- желание <<выпендриться>>);
		\item проектные требования,  продиктованные текущей модой,  маркетингом или «левой пяткой заказчика»;
	\end{itemize}
\end{frame}

\begin{frame}{6. Правило расчетливости}
	\begin{block}{Пишите большие программы,  только если после демонстрации становится ясно,  что ничего другого не остается.}
		Под <<большими программами>> здесь понимаются программы с большим объемом кода и значительной внутренней сложностью.
	\end{block}
\end{frame}

\begin{frame}{7. Правило прозрачности}
	\begin{block}{Для того,  чтобы упростить проверку и отладку программы,  ее конструкция должна быть обозримой.}
		Программа {\itshape прозрачна}, если при ее минимальном изучении можно понять, что она делает и как.\\
		Программа {\itshape воспринимаема},  когда она имеет средства для мониторинга и отображения внутреннего состояния.
	\end{block}
	Необходимо использовать достаточно простые форматы входных и выходных данных.\\
	Интерфейс должен быть приспособлен для использования в отладочных сценариях.
\end{frame}

\begin{frame}{8. Правило устойчивости}
	\begin{block}{Устойчивость -- следствие	прозрачности и простоты.}
		Программа является {\itshape устойчивой},  когда она выполняет свои функции в неожиданных условиях,  которые выходят за рамки предположений разработчика,  как и в нормальных условиях.\\
		Программа является {\itshape простой},  если происходящее в ней не представляется сложным для восприятия человеком.
	\end{block}
	Один из способов организации -- модульность(простые блоки,  ясные интерфейсы)\\
	Следует избегать частных случаев!
\end{frame}

\begin{frame}{Пример неусточивого ПО}
	\begin{center}
		\includegraphics[width=1\textwidth]{../../slides/intro/exploits_of_a_mom_rus.png}
	\end{center}
\end{frame}


\begin{frame}[fragile]{Пример <<простой>> программы}
	\begin{center}
		\begin{verbatim}
+++++++++++++++++++++++++++++++++++++++++++++
+++++++++++++++++++++++++++.+++++++++++++++++
++++++++++++.+++++++..+++.-------------------
---------------------------------------------
---------------.+++++++++++++++++++++++++++++
++++++++++++++++++++++++++.++++++++++++++++++
++++++.+++.------.--------.------------------
---------------------------------------------
----.-----------------------.
		\end{verbatim}
	\end{center}
\end{frame}

\begin{frame}{9. Правило представления}
	\begin{block}{Знания следует оставлять в данных,  чтобы логика программы могла быть примитивной и устойчивой.}
		Даже простую логику бывает сложно проверить,  но даже сложные структуры данных являются довольно простыми для моделирования и анализа (например диаграмма 50 узлов дерева и блок-схема 50 строк кода)
	\end{block}
	Если можно выбирать между усложнением структуры данных и усложнением кода,  то лучше выбирать первое.\\
	Примеры: ascii,  генератор html-таблицы.
\end{frame}

\begin{frame}{10. Правило наименьшего удивления}
	\begin{block}{При проектировании интерфейсов всегда следует использовать наименее неожиданные элементы.}
		Необходимо учитывать характер предполагаемой аудитории и традиции платформы.
	\end{block}
	Оборотная сторона: следует избегать создания внешне похожих вещей,  слегка отличающихся в действительности,  поскольку {\itshape кажущаяся привычность порождает ложные ожидания}.
\end{frame}

\begin{frame}{11. Правило тишины}
	\begin{block}{Если программе нечего сказать,  то пусть лучше молчит.}
		Внимание и сосредоточенность пользователя -- ценный и ограниченный ресурс,  который требуется только в случае необходимости.
	\end{block}
	Важная информация не должна смешиваться с подробными сведениями о работе программы.
\end{frame}

\begin{frame}{12. Правило восстановления}
	\begin{block}{Когда программа завершается аварийно,  это должно происходить явно (шумно) и по возможности быстро.}
		Если программа не способна справиться с ошибкой,  то необходимо завершить ее работу так,  чтобы максимально упростить диагностику.
	\end{block}
	Для сетевых служб следует следовать рекомендации Постела:\\
	<<{\itshape Будьте либеральны к тому,  что принимаете,  и консервативны к тому,  что отправляете}>>
\end{frame}

\begin{frame}{13. Правило экономии}
	\begin{block}{Время программиста дорого -- поэтому задача экономии его времени более приоритетна,  по сравнению с экономией машинного времени.}
		Компьютер железный -- ему не скучно (с) программистская мудрость
	\end{block}
	Использование высокоуровневых языков и <<обучение>> машины выполнять больше низкоуровневой работы по программированию,  что приводит к правилу 14.
\end{frame}

\begin{frame}{14. Правило генерации}
	\begin{block}{Избегайте кодирования вручную; если есть возможность -- пишите программы для создания программ.}
		Использование генераторов кода оправданно,  когда они могут повысить уровень абстракции,  
		т.е. когда язык спецификации для генератора проще,  чем сгенерированный код,  
		и код впоследствии не потребует ручной доработки.
	\end{block}
	Примеры: грамматические и лексические анализаторы,  генераторы make-файлов,  построители GUI-интерфейсов.
\end{frame}

\begin{frame}{15. Правило оптимизации}
	\begin{block}{Сначала -- опытный образец,  потом -- оптимизирование.}
		Добейтесь стабильной работы,  только потом оптимизируйте.
	\end{block}
	\begin{block}{Керниган и Плоджер:}
		90\% актуальной и реальной функциональности лучше,  чем 100\% функциональности перспективной и сомнительной
	\end{block}
\end{frame}

\begin{frame}{15. Правило оптимизации}
	\begin{block}{Кнут:}
		преждевременная оптимизация -- корень всех зол
	\end{block}
	\begin{block}{Кент Бек (экстремальное программирование):}
		заставьте программу работать,  заставьте работать ее верно,  а затем сделайте ее быстрой
	\end{block}
\end{frame}

\begin{frame}{16. Правило разнообразия}
	\begin{block}{Не следует доверять утверждениям о <<единственно правильном пути>>.}
		Никто не обладает умом,  достаточым для оптимизации всего или для предвидения всех возможных вариантов использования создаваемой программы.
	\end{block}
\end{frame}

\begin{frame}{17. Правило расширяемости}
	\begin{block}{Разрабатывайте для будущего. Оно наступит быстрее,  чем вы думаете.}
		При проектировании протоколов или форматов файлов следует делать их самоописательными,  для того,  чтобы их можно было расширить.
	\end{block}
	{\itshape Всегда},  следует либо включать номер версии,  либо составлять формат из самодостаточных,  
	самоописательных команд так,  чтобы можно было легко добавить новые директивы,  
	а старые удалить, <<не сбивая с толку>> код чтения формата.
\end{frame}

\begin{frame}{Все правила сразу}
	\begin{center}
	{\Huge\bfseries K.I.S.S.}

	Keep It Simple,  Stupid!
	\end{center}
\end{frame}


}

\begin{frame}{Источники информации о UNIX way}
  \begin{itemize}
    \item Эрик С. Раймонд (ESR) Искусство программирования для UNIX
    \item Joel Spolsky. Biculturalism \url{https://www.joelonsoftware.com/2003/12/14/biculturalism/}
    \item \url{https://suckless.org} -- несколько экстремистский сайт
  \end{itemize}
\end{frame}

\section{Оболочка командной строки: основные принципы}

\mode<all>{\input{../../slides/cmdline/clui_vs_gui.tex}}

\mode<all>{\begin{frame}[fragile]{Оболочка операционной системы}

     \begin{block}{Оболочка операционной системы}
     (от англ. shell «оболочка») \alert{интерпретатор команд} операционной системы, обеспечивающий интерфейс для взаимодействия пользователя с функциями системы.  
     \end{block}
     \only<1>{
     Что такое Unix shell?
     \begin{itemize}
        \item Обычная программа, запускающаяся после входа в систему
        \item Интерактивный командный интерпретатор
        \item Платформа интеграции (для утилит)
        \item Язык программирования
     \end{itemize}
   }
   \only<2>{
	\vspace{0.5in}
	Пример shell из  Windows-world --  cmd.exe, PowerShell
	\vspace{0.5in}
	Минимальный дистрибутив Linux -- ядро + shell 
   }
\end{frame}
\note{   
     Как вы помните задачи ядра - управление процессами и потоками, управление
     памятью, Управление файлами, Разграничение доступа, мониторинг и
     конфигурация. Shell обеспечивает пользователю  интерфейс доступа к этим функциям ядра.    
     Как Интерактивный командный интерпретатор shell предосталяет средства:
     редактора командной строки, часто со средствами упрощающими ввод команд
     такими как  горячие клавишами, автодополнение, история команд и поиск по
     истории, сокращения команд (aliases), сокращения путей к файлам c помощью
     спецсимволов. 
     Как Платформа интеграции предоставляет средства пренаправления
     ввода/вывода, каналы (pipes), и возможность создавать скрипты.
     Одновременно является языком программирования, но эту часть будем
     рассматривать позже. 
}


\begin{frame}[fragile]{Как работать с командной строкой}
  \begin{itemize}
    \item 
	Находим приглашение командной строки
	\$, \#, user@host:~\$
    \item
	Вводим имя команды, опции, аргументы, запускаем на выполнение нажатием <Enter>
   \end{itemize}

	Что такое команды?
  \begin{itemize}
    \item исполняемая программа (бинарный файл, скрипт)
    \item встроенные в оболочку команды (shell built-ins)
    \item функция оболочки
    \item сокращение команды (an alias) 
  \end{itemize}
\end{frame}
\note {
Система ожидает ввода команды, показывая приглашение командной строки.
Пользователь вводит команды опции аргументы и нажимает <Enter>. 
Опции - модификаторы поведения программы.  
Аргументы - то над чем производятся действия.
}

}

\subsection{Получение помощи}

\mode<all>{
\begin{frame}[fragile]{Как правильно задавать вопросы}
From FAQ How To Ask Questions The Smart Way
Before You Ask
  \begin{itemize}
	  \item Try to find an answer by reading the manual.
	  \item Try to find an answer by reading a FAQ.
	  \item Try to find an answer by searching the archives of the forum you plan to post to.
	  \item Try to find an answer by searching the Web.
	  \item Try to find an answer by inspection or experimentation.
	  \item Try to find an answer by asking a skilled friend.
	  \item If you're a programmer, try to find an answer by reading the source code.
    \end{itemize}
\end{frame}


\begin{frame}[fragile]{Источники получение помощи}
  \begin{itemize}
    \pause
    \item \textbf{man} - помощь по внешним командам
    \pause
    \item \textbf{help} - встроенная помощь по внутренним командам bash (также man bash)
    \pause
    \item \textbf{info} - расширенная помощь по некоторым командам (texinfo format)
  \end{itemize}
     \begin{block}{Упражнение. Работа с командой info}
        \begin{itemize}
        \item   Попробовать {\tt info coreutils}
        \item   Справка по навигации -- нажать h
        \end{itemize}
	 \end{block}
\end{frame}
\begin{frame}[fragile]{Источники помощи}
		\begin{block}{Упражнение. Другие источники помощи.}
			\begin{lstlisting}
            help 
            help help
            man -h
            info --help
			\end{lstlisting}
		\end{block}
\end{frame}

\begin{frame}[fragile]{Основное о man}
\begin{columns}
	\column{2.5in}
		\begin{itemize}
			\item Прочитайте {\tt man man} !
			\item apropos, аналог {\tt man -k <слово>}
            \item whatis, аналог {\tt man -f <слово>} 
			\item Разделы (sections)
				\begin{itemize}
					\item[1] Основная секция(юзерские программы)
					\item[2] Syscalls
					\item[3] С library
					\item[5] Конфигурационные файлы
					\item[8] Системные службы
				\end{itemize}
		\end{itemize}
	  \textbf{Замечание}

	  Обычно внутри страницы работает поиск с помощью '/'
	\pause 
	
	\column{1in}
		\begin{block}{Попробовать}
			\begin{lstlisting}
man -k intro or apropos
man -f intro or whatis
man 3 intro
man 1 intro
man -wa intro
			\end{lstlisting}
		\end{block}
	\end{columns}
\end{frame}


%\begin{frame}[fragile]{Чему научились}
%  \begin{itemize}
%  \item Как спрашивать у сообщества
%  \item Умеем использовать 3 источника получения информации man, info, help
%  \item Как перемещаться по страницам помощи info и man
%  \item Иcкать в системе помощи man и запрашивать из одного из 8-ми разделов 
%  \end{itemize}
%\end{frame}
}

\subsection{Процессы}

\mode<all>{\input{../../slides/cmdline/process-intro.tex}}

\mode<all>{\input{../../slides/cmdline/pipes.tex}}
\mode<all>{\input{../../slides/cmdline/io-redirection.tex}}

\mode<all>{\input{../../slides/cmdline/io-redirection-here.tex}}
\mode<all>{\input{../../slides/cmdline/commands1.tex}}



\end{document}

